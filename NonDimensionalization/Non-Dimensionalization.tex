\documentclass[a4paper,11pt]{article}
\usepackage{xeCJK}
\setCJKmainfont{KaiTi}

\title{Non-Dimensionalization}
\author{Li Ji}
\usepackage{xeCJK}
\setCJKmainfont{KaiTi}
\setCJKsansfont{KaiTi}
\setCJKmonofont{KaiTi}
\usepackage{wallpaper}
\usepackage{subfigure}
\usepackage{lineno,hyperref}
\usepackage{indentfirst}
\usepackage{bm}
\usepackage{threeparttable}
\usepackage{array}
\usepackage{multirow}
\usepackage{amsmath}
\usepackage{amssymb}
\usepackage{graphicx}
\usepackage{url}
\usepackage{hyperref}
\usepackage{epstopdf}

\begin{document}

\maketitle
\tableofcontents

\begin{abstract}
\end{abstract}

\section{Basic Non-Dimensionalization System}
说明1:本无量纲系统采用与SU2一致的无量纲方式!其目的是为了保证无量纲化前后计算公式的一致性; \\
说明2:设$param$为某参数,则$\overline{param}$则表示该参数的无量纲值,$param_{ref}$表示该参数的参考值;\\
说明3:$param=\overline{param} \times param_{ref}$。
\begin {enumerate}
    \item {Basic Reference Parameters} \\
        本无量纲系统中有四个基本的参考参数:\\
        $p_{ref}$、$T_{ref}$、 $\rho_{ref}$、$L_{ref}$ \\
        其余参考参数都由以上四个基本参考值确定。\\
        众所周知,理想气体的状态方程为:
        \begin{equation}
            p=\rho R T
        \end{equation}
        而为了保持无量纲化前后计算公式的一致性,则无量纲的理想气体状态方程应为:
        \begin{equation}
            \overline{p}=\overline{\rho} \overline{R} \overline{T}
        \end{equation}
        这里我们引进气体常数$R$的参考值$R_{ref}$。
        \begin{equation}\label{refrelationSU2}
            p=\rho R T \Rightarrow \overline{p}p_{ref}=\overline{\rho}\rho_{ref}\overline{R}R_{ref}\overline{T}T_{ref} \\
            \Rightarrow R_{ref} = \frac{p_{ref}}{\rho_{ref}T_{ref}}
        \end{equation}
        式.\eqref{refrelationSU2} 是SU2量纲系统中必须满足的条件。
        在GKS中: $\lambda=\frac{1}{2RT}$, 所以
        \begin{equation} \label{plambda1}
            p = \frac{\rho}{2\lambda}
        \end{equation}
        为了保持公式的不变性,无量纲后
        \begin{equation} \label{plambda2}
            \overline{p} = \frac{\overline{\rho}}{2\overline{\lambda}}
        \end{equation}
        结合式\eqref{refrelationSU2}、\eqref{plambda1} 与 \eqref{plambda2}可得
        \begin{equation}\label{refrelationGKS}
            \lambda_{ref} = \frac{\rho_{ref}}{p_{ref}}
        \end{equation}
        式\eqref{refrelationGKS}为GKS必须要满足的条件。由此可得
        \begin{equation}
            \overline{\lambda} = \frac{1}{2\overline{R}\overline{T}}
        \end{equation}

    \item {$u_{ref}$} \\
        这里我们推导参考速度$u_{ref}$ \\
        声速的计算公式为
        \begin{equation}
            a=\sqrt{\frac{\gamma}{2\lambda}}
        \end{equation}
        为了保持计算公式的一致性,无量纲化后应有:
        \begin{equation}
            \overline{a}=\sqrt{\frac{\gamma}{2\overline{\lambda}}}
        \end{equation}
        采用与以上相同的过程,可得
        \begin{equation}
            u_{ref} = \sqrt{\frac{1}{\lambda_{ref}}}=\sqrt{\frac{p_{ref}}{\rho_{ref}}}
        \end{equation}
    \item {$t_{ref}$}\\
        显而易见,参考时间为
        \begin{equation}
            t_{ref} = \frac{L_{ref}}{u_{ref}}
        \end{equation}
\end{enumerate}

\section{Cases}
无量纲系统的意义在于消除了不同单位制的差异,以及不同无量纲系统之间的转化,现举例如下:
\begin{enumerate}
    \item {Case1} \\
        \begin{eqnarray}
            p_{ref} = 2p_{\infty} \\
            \rho_{ref} = \rho_{\infty} \\
            T_{ref} = T_{\infty} \\
            L_{ref} = 1 \\
        \end{eqnarray}
        四种基本量纲都已给出。

    \item {Case2} \\
        \begin{eqnarray}
            p_{ref} = p_{\infty} \\
            \rho_{ref} = \rho_{\infty} \\
            T_{ref} = T_{\infty} \\
            L_{ref} = 1 \\
        \end{eqnarray}
        四种基本量纲都已给出。
    \item {Case3} \\
        \begin{eqnarray}
            u_{ref} = a_{\infty} \\
            \rho_{ref} = \frac{\rho_{\infty}}{\gamma} \\
            L_{ref} = 1 \\
            R_{ref} = R_{\infty}
        \end{eqnarray}
        在此量纲系统中$T_{ref}$与$p_{ref}$被$u_{ref}$与$R_{ref}$所取代,所以我们需要求出
        $T_{ref}$与$p_{ref}$
        因为
        \begin{equation}
            u_{ref} = \sqrt{\frac{p_{ref}}{\rho_{ref}}}
        \end{equation}
        所以可得
        \begin{equation}
            p_{ref} = \rho_{ref}u_{ref}^2 = \frac{\rho_{\infty}}{\gamma}\gamma R_{\infty}T_{\infty}\\
            =\rho_{\infty}R_{\infty}T_{\infty}
        \end{equation}
        \begin{equation}
            T_{ref}=\frac{p_{ref}}{\rho_{ref}R_{ref}}=\frac{\rho_{\infty}R_{\infty}T_{\infty}}{\frac{\rho_{\infty}}{\gamma}R_{\infty}}
            =\gamma T_{\infty}
        \end{equation}
        至此,四个基本参考参数都已求出,其余参考参数可用基本参考值表示。易验证,两种量纲系统统一无冲突。
    \item {Case4} \\
        \begin{eqnarray}
            \rho_{ref} = \rho_{\infty} \\
            \lambda_{ref} = \lambda_{\infty} \\
            L_{ref} = 1 \\
            R_{ref} = R_{\infty}
        \end{eqnarray}
        在此量纲系统中$p_{ref}$与$T_{ref}$被$\lambda_{ref}$与$R_{ref}$所取代,所以我们需要求出
        $p_{ref}$与$T_{ref}$,其过程如下:
        \begin{eqnarray}
            T_{ref}=\frac{1}{R_{ref}\lambda_{ref}}=\frac{1}{R_{\infty}\lambda_{\infty}} \\
            = \frac{1}{R_{\infty}\frac{1}{2R_{\infty}T_{\infty}}}=2T_{\infty}\\
            p_{ref}=\rho_{ref}R_{ref}T_{ref}=2\rho_{\infty}R_{\infty}T_{\infty}
        \end{eqnarray}
        至此,四个基本参考参数都已求出,其余参考参数可用基本参考值表示。易验证,两种量纲系统统一无冲突。

\end{enumerate}
\end{document}
