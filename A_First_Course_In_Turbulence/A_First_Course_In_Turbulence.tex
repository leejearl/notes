\documentclass[review]{elsarticle}
\usepackage{natbib}
\usepackage{graphicx, subfigure}
\usepackage{dcolumn}% Align table columns on decimal point
\usepackage{bm}% bold math

\usepackage{lineno,hyperref, url}
\usepackage{indentfirst}
\usepackage{array, threeparttable}
\usepackage{multirow}
\usepackage{mathrsfs}
\usepackage{color}
%\usepackage{amsmath, amssymb}

%\numberwithin{equation}{section}

\modulolinenumbers[5]

\journal{Journal of \LaTeX\ Templates}

%%%%%%%%%%%%%%%%%%%%%%%
%% Elsevier bibliography styles
%%%%%%%%%%%%%%%%%%%%%%%
%% To change the style, put a % in front of the second line of the current style and
%% remove the % from the second line of the style you would like to use.
%%%%%%%%%%%%%%%%%%%%%%%

%% Numbered
%\bibliographystyle{model1-num-names}

%% Numbered without titles
%\bibliographystyle{model1a-num-names}

%% Harvard
%\bibliographystyle{model2-names.bst}\biboptions{authoryear}

%% Vancouver numbered
%\usepackage{numcompress}\bibliographystyle{model3-num-names}

%% Vancouver name/year
%\usepackage{numcompress}\bibliographystyle{model4-names}\biboptions{authoryear}

%% APA style
%\bibliographystyle{model5-names}\biboptions{authoryear}

%% AMA style
%\usepackage{numcompress}\bibliographystyle{model6-num-names}

%% `Elsevier LaTeX' style
\bibliographystyle{elsarticle-num}
%%%%%%%%%%%%%%%%%%%%%%%

\begin{document}

\begin{frontmatter}
    \title{The Notes of \textit{A FIRST COURSE IN TURBULENCE}}
    \author{Ji Li}
    \ead{leejearl@mail.nwpu.edu.cn}


    \address{National Key Laboratory of Science and Technology on Aerodynamic Design and Research, Northwestern Polytechnical 
    University, Xi'an, Shaanxi 710072, China}

    \begin{abstract}
        The Notes of \textit{A FIRST COURSE IN TURBULENCE}
    \end{abstract}
\end{frontmatter}

\section{Scale Relations}

\section{Equations}

\section{Issues}
\subsection{Issues1}
\begin{itemize}
    \item \textbf{Introduction}
    \item \textbf{Hypothesis}
    \item \textbf{Equations and Scale Relations}
    \item \textbf{Summary}
\end{itemize}

\section{Introduction}
	\subsection{The nature of turbulence}
		\subsubsection {Irregularity}
		\subsubsection {Diffusivity}
		\subsubsection {Large Reynolds numbers}
		\subsubsection {Three dimensional vorticity fluctuations}
		\subsubsection {Disspation}
		\subsubsection {Continuum}
		\subsubsection {Turbulent flows are flows}
			Turbulence is not a feature of fluids but of fluid flows.
	\subsection{Methods of analysis}
		\subsubsection {Dimensional analysis}
		\subsubsection {Asymptotic invariance}
		\subsubsection {Local invariance}
	\subsection{The Origin of turbulence}
	\subsection{Diffusivity of turbulence}
		The ability of turbulence transferring and mixing , which is the outstanding characteristic of turbulence motion, 
		is stronger than molecular diffusion.
		\subsubsection{Diffusion in a problem with an imposed length scale}
			Diffusion of laminar flow and turbulent flow are discussed in this section with an imposed length scale.
			Diffusion equation of temperature read as
			\begin{equation}\label{1_diffusionEquation}
				\frac{\partial \theta}{\partial {t}}=\gamma \frac{\partial ^2\theta}{\partial x_i\partial x_i}.
			\end{equation}
			Where $\theta$ is the temperature and $\gamma$ is the thermal diffusivity. Since $\nu / \gamma = 0.73$, $\gamma$ 
			can be estimated as $\nu$. $\nu$ is the kinematic viscosity.
			Relation used in this section is
			\begin{equation}
				\gamma \sim \nu
			\end{equation}
			Dimensionally, Eq. \ref{1_diffusionEquation} can be interpreted as
			\begin{equation}
				\frac{\Delta \theta}{T_m} \sim \gamma \frac{\Delta \theta}{L^2} \Rightarrow T_m \sim \frac{L^2}{\gamma}.
			\end{equation}
			Where $T_m$ denotes the time scale of molecular motion and $L$ is the length scale.
			For turbulent flow with length scale $L$, 
			\begin{equation}
				T_t = \frac{L}{u}.
			\end{equation}
			Where $T_t$ means the time scale of turbulence and $u$ is the velocity scale of velocity fluctuations in turbulence.
			Therefore, the relation of $T_m$ and $T_t$ can be written as
			\begin{equation}\label{TtTm}
				\frac{T_t}{T_m} \sim \frac{\gamma}{Lu} \sim \frac{\nu}{Lu} \sim \frac{1}{R_e}.
			\end{equation}
			Eq. \ref{TtTm} shows that Reynolds number of a turbulent flow can be interpreted as ratio of a turbulence time scale to 
			a molecular time scale.
			\begin{center}
				\fbox{%  
					\parbox{\textwidth}{%  
						\centering {\textbf{Remarks}}
						\begin{itemize}
							\item $\gamma$ is of order $\nu$.
							\item $R_e$ in Eq. \ref{TtTm} is often very large. 
							\item With imposed length scale, Eq. \ref{TtTm} can be interpreted as time scale of molecular diffusion $T_m$ is 
								larger than the time scale of turbulent flow $T_t$. So the rate of transfer or mixing momentum of turbulent 
								motion is much greater than molecular diffusion.  
						\end{itemize}
			    	}  
				}%  			
			\end{center}


		\subsubsection 	{Eddy diffusivity}
			If we use an eddy diffusivity concept. Diffusion equation can be written as
			\begin{equation}
				\frac{\partial \theta}{\partial {t}}=K \frac{\partial ^2\theta}{\partial x_i\partial x_i}.
			\end{equation} 
			Where $K$ is the representation diffusivity(often called "eddy" diffusivity). Therefore, 
			\begin{equation}
				\frac{L^2}{K} \sim T = T_t \sim \frac{L}{u} \Rightarrow K \sim Lu.
			\end{equation}
			The ratio of $K$ to $\gamma$ is
			\begin{equation}
				\frac{K}{\gamma} \sim \frac{K}{\nu}=\frac{Lu}{\nu}=R_e.
			\end{equation}
			\begin{center}
				\fbox{%  
					\parbox{\textwidth}{%  
						\centering {\textbf{Remarks}}
						\begin{itemize}
							\item Since $Re$ is large, $K$ is often very larger than $\gamma$.
							\item $K$ can be interpreted as apparent viscosity, and $\gamma$ is of order $\nu$.
						\end{itemize}
			    	}  
				}%  			
			\end{center}			
		\subsubsection{Diffusion in a problem with imposed time scale}
			With a imposed time scale, Eq. \ref{1_diffusionEquation} may be interpreted as
			\begin{equation} 
				T \sim \frac{L_m^2}{\gamma}.
			\end{equation}
			With the relation
			\begin{equation}
				T \sim \frac{L_t}{u},
			\end{equation}
			the ratio of $L_t$ and $L_m$ can be read as
			\begin{equation}
				\frac{L_t}{u} \sim \frac{L_m^2}{\gamma} \Rightarrow \frac{L_t}{L_m} = R_e^{\frac{1}{2}}.
			\end{equation}
	\subsection {Length scale in turbulent flows}
		\subsubsection{Laminar boundary laryer}
			For steady flow of an incompressible fluid with constant viscosity, the Navier-Stokes equations are
			\begin{equation}\label{NS1}
				u_j\frac{\partial u_i}{\partial x_j} = -\frac{1}{\rho}\frac{\partial p}{\partial x_i} 
					+ \nu \frac{\partial ^2u_i}{\partial x_j\partial x_j}.			
			\end{equation}
			If we estimate the inertia terms as $U^2/L$ and the viscous term as $\nu U/L^2$, the ratio of the two 
			is $UL/\nu$. It indicates that the viscous term could be negligible at large Reynolds number. In fact, 
			it is impossible to neglect viscous terms everywhere in the flow filed. To retain the viscous terms, a 
			different length scale $l$ must be selected. With such a length scale, the relation of inertia terms and 
			viscous terms reads
			\begin{equation}
				\nu U/L^2 \sim UL/\nu \Rightarrow \frac{l}{L} \sim \left( \frac{\nu}{UL}\right)^{1/2}=R^{-1/2}_e. 
			\end{equation}
			The viscous length $l$ is a transverse length scale, and it represents the width of boundary layer.
		\subsubsection{Diffusive and convective length scale}
			Taking fluid flow over flat plate as the example, the boundary layer is much smaller than the scale $L$ 
			of the flow field. $L$ is the length scale of convection along the flow and $l$ is the length scale of 
			diffusion across the flow. 
		\subsubsection{Turbulent boundary layer}
			The turbulent eddies transfer momentum deficit away from the surface. The length scale of turbulent diffusion 
			is $l$. The velocity scale of velocity fluctuations is $u$. The length scale and velocity scale of flow field 
			are $L$ and $U$. Equating the time scale of diffusion and flow field, wen can write the scale relations for 
			boundary layers as
			\begin{equation}
				l/u \sim L/U \Rightarrow l/L \sim u/U.
			\end{equation}
			\begin{center}
				\fbox{%  
					\parbox{\textwidth}{%  
						\centering {\textbf{Remarks}}
						\begin{itemize}
							\item In this section, the length scale of turbulent eddies is the length scale of larger eddies.
							\item The time scale and length scale are not unique in turbulent flow.
						\end{itemize}
			    	}  
				}%  			
			\end{center}
						
		\subsubsection{Laminar and turbulent friction}
			For a steady laminar boundary layer in two-dimensional flow on a plat with length $L$, the drag $D$ per unit span 
			is equal to the total rate of loss of momentum. The mass flow deficit at the end of the plate is proportional to 
			$\rho UL$. So, we may put 
			\begin{equation} 
				D \sim \rho U^2L \Rightarrow C_d \sim \frac{D}{\frac{1}{2}\rho U^2L}=2\frac{l}{L}=2R_e^{-1/2}.
			\end{equation} 
			For a turbulent flow, the mass flow deficit at the end of the plate is proportional to $\rho ul$, so that the rate of 
			loss of momentum is proportional to $(\rho ul)U$. Consequently,
			\begin{equation}
				D \sim \rho uUl \Rightarrow C_d \sim \frac{D}{\frac{1}{2}\rho U^2L}=2\frac{l}{L}\frac{u}{U}=2\left( \frac{u}{U}\right)^2.
			\end{equation}	 
			Experiment evidence shows that the turbulence level $u/U$ varies very slowly with Reynolds number, so that the drag 
			coefficient of a tubulent boundary should be very much larger than the drag coefficient of a laminar boundary layer.
		\subsubsection{Small scales in turbulence}
			We have suggested that the large eddies scale are as big as the width of flow. The Kolmogorov length and time scales 
			are the smallest scales occurring in turbulent motion. The Kolmogorov scales are
			\begin{equation}\label{KolmogorovScales}
				\eta \equiv \left( \nu ^3 / \epsilon \right)^{1/4}, \tau \equiv \left( \nu / \epsilon \right)^{1/2},
				\upsilon \equiv \left( \nu \epsilon \right)^{1/4}.
			\end{equation}
			The Reynolds number which defined with Eq. \ref{KolmogorovScales} is
			\begin{equation}\label{KolmogorovRe}
				R_e = \frac{\eta\upsilon}{\nu} \equiv 1.
			\end{equation}
			Eq. \ref{KolmogorovRe} shows that the small-scale motion is quite viscous.
	 
			
		\subsubsection{\underline{A inviscid estimate for the dissipation rate}}
			To be continued.
		\subsubsection{Scale relations}
			The scales of smallest eddies are very mush smaller than those of largest eddies. Most of the energy is associated with 
			large-scale motions, and most of the vorticity is associated with small-scale motions.
		\subsubsection{Molecular and turbulent scales}
			The scales of molecular motion are shown in Table \ref{scales}.
			
\section{Turbulent transport of momentum and heat}
	Turbulent velocity fluctuations can generate large momentum fluxes between different parts of a flow. A momentum flux can 
	be thought of as a stress.
	\subsection{The Reynolds equations}	
	\subsubsection{The Reynolds decompositions}
		The mean value of fluctuating quantity itself is zero by definition; for example,
		\begin{equation}\label{meanfluctuating}
			\bar{u_i} = \lim _{T \rightarrow \infty} \frac{1}{T} \int_{t_0}^{t_0 + T} (\tilde{u_i} - U_i)dt \equiv 0.
		\end{equation}
		For a time average to make sense, the integrals in Eq. \ref{meanfluctuating} has to be independent of $t_0$. In 
		other words, the mean flow has to be steady.
		\begin{equation}
			\frac{\partial U_i}{\partial t} = 0.
		\end{equation}
		
	\subsubsection{Correlated variables}
		The correlation coefficient is defined as 
		\begin{equation} \label{cij}
			c_{ij} = \overline{u_i u_j}/\left( \overline{u_i^2} \cdotp \overline{u_j^2}\right)^{1/2}.
		\end{equation}
	\subsubsection{Equations for mean flow}
		The equations of motion for mean flow read as
		\begin{equation}
			U_j\frac{\partial U_i}{\partial x_j} + \overline{u_j\frac{\partial u_i}{\partial x_j}} = \frac{1}{\rho}\frac{\partial}{\partial x_j}{\sum}_{ij}.
		\end{equation}
		We may write 
		\begin{equation}\label{turbulent-momentum-transfer}
			\overline{u_j\frac{\partial u_i}{\partial x_j}} = \frac{\partial}{\partial x_j}\overline{u_iu_j}.
		\end{equation}
		Eq. \ref{turbulent-momentum-transfer} represents the mean transport of fluctuating momentum by turbulent velocity fluctuations.
		Reynolds momentum equation reads as
		\begin{equation} \label{ReMomEq}
			U_j\frac{\partial U_i}{\partial x_j} = \frac{1}{\rho}\frac{\partial}{\partial x_j}({\sum}_{ij} - \overline{\rho u_iu_j}).
		\end{equation}
	\subsubsection{The Reynolds stress}
		The contribution of the turbulent motion to the mean stress tensor is designated by the symbol $\tau_{ij}$:
		\begin{equation}
			\tau_{ij} = - \overline{\rho u_iu_j}.
		\end{equation}
	\subsubsection{\underline{Turbulent transport of heat}}
		To be continued.
	\subsection{Elements of the kinetic theory of gases}
	\subsubsection{Pure shear flow}
		Let us take a steady pure shear flow, homogeneous in the $x_1$, $x_3$ plane. The only nonvanishing velocity 
		component is take to be $U_1$; it is a function of $x_2$ only. If the flow is laminar, the only nonvanishing 
		components of the viscous shear stress are 
		\begin{equation}
			\sigma_{12} = \sigma_{21} = \mu \partial U_1 / \partial x_2.
		\end{equation}
		The shear stress $\sigma_{12}$ can be resulted from the molecular transport of momentum in the $x_2$ direction.
		Let $v_1$ and $v_2$ be the $x_1$ and $x_2$ components of the instantaneous velocity of a molecular relative to 
		the mean flow. $m$ represents the mass of a molecular. $N$ denotes the number of molecular per unit volume. Now, 
		$Nm$ is the mass per unit volume $\rho$. The transport of momentum $mv_1$ of a molecular in direction $x_2$ with 
		velocity $v_2$ reads
		\begin{equation}
			\sigma_{12} = -Nm\overline{v_1v_2} = -\rho\overline{v_1v_2}.
		\end{equation}

	\subsubsection{Molecular collisions}
		Suppose the mean free path is $\xi$. If a molecular traveling from position $x_2 = -\xi$ to $x_2 = 0$, it brings 
		a momentum deficit 
		\begin{equation}\label{MomDeficit1}
			M = m[U_1(0) - U_1(-\xi)] \approx m\xi\frac{\partial U_1}{\partial x_2}+\frac{1}{2}m\xi^2\frac{\partial^2U_1}{\partial x_2^2}.
		\end{equation}  
		A local length scale $l$ of the flow $U_1(x_2)$ is defined as
		\begin{equation}\label{locallengthscale}
			l \equiv \frac{\partial U_1/\partial x_2}{\partial^2U_1/\partial x_2^2}.
		\end{equation}
		If $l \gg \frac{1}{2}\xi$, Eq. \ref{MomDeficit1} may be approximate by
		\begin{equation}
			M = m\xi\frac{\partial U_1}{\partial x_2}.
		\end{equation}
		We can write
		\begin{equation}
			\sigma_{12} = \alpha MNa = \alpha Nma\xi\partial U_1/\partial x_2,
		\end{equation}
		where $a$ represents the sound speed.
	\subsubsection{\underline{Characteristic times and lengths}}
		To be continued.
	\subsubsection{\underline{The correlation between $v_1$ and $v_2$}}
		To be continued.
	\subsubsection{\underline{Thermal diffusivity}}
		To be continued.
	
	\subsection{Estimates of the Reynolds stress}
	\subsubsection{Reynolds stress and vortex stretching}
		\begin{itemize}
			\item The existence of a Reynolds stress requires that the velocity fluctuations $u_1$ and $u_2$ be correlated.
			\item Because eddies are continuously losing energy to smaller eddies, they need shear to maintain their energy.
			\item The most powerful eddies are those that can absorb energy from the shear flow more effectively than others.
			\item \underline{To be continued.}
		\end{itemize}
	\subsubsection{The mixing-length model}
		Using the Reynolds decomposition of velocities, we can write the momentum deficit as 
		\begin{equation}\label{MomDeficit2}
			\Delta M = \rho [U_1(x_2)-U_1(0)] + \rho[u_1(x_2,t)-u_1(0,0)].
		\end{equation}
		If the contribution of the turbulence to the momentum deficit can be neglected and if the difference $U_1(x_2)-U_1(0)$ 
		may be approximated by $x_2\partial U_1/\partial x_2$, where the gradient is taken at $x_2 = 0$, $\Delta M$ may be 
		approximated by 
		\begin{equation}
			\Delta M = \rho x_2\partial U_1/\partial x_2.
		\end{equation}  
		The volume transported per unit area and unit time in the $x_2$ direction is $\tilde{u}_2$ of the moving point. Now 
		$\tilde{u}_2=dx_2/dt$, so that the average momentum flux at $x_2=0$ may be written as
		\begin{equation}
			\tau_{12} = \rho x_2 \frac{\partial U_1}{\partial x_2}\overline{\tilde{u}_2} = 
			 \rho x_2 \frac{\partial U_1}{\partial x_2}\frac{d\overline{x}_2}{dt} = 
			 \frac{1}{2}\rho \frac{\partial U_1}{\partial x_2}\frac{d\overline{x_2^2}}{dt}.
		\end{equation}
		If we assume that $u_2$ and $x_2$ become essentially uncorrelated at values of $x_2$ comparable to some transverse 
		length scale $l$, we may estimate that $\overline{x_2u_2}$ is of order $u_2'l$. Here $u_2'$ is the rms velocity in 
		the $x_2$ direction; the dispersion length scale $l$ is called the \textit{mixing length}.
	\subsubsection{\underline{The length-scale problem}}
		To be continued.
	\subsubsection{\underline{A neglected transport term}}
		To be continued.
	\subsubsection{The mixing length as an integral scale}
	\subsubsection{The gradient-transport fallacy}
	\subsubsection{Further estimates}
	\subsection{Turbulent heat transfer}

\section{The dynamics of turbulence}
	\subsection{Kinetic energy of mean flow}
		The energy equation reads
		\begin{equation}\label{enerEqMeanFlow1}
			\underbrace {\rho U_j\frac{\partial}{\partial x_j}(\frac{1}{2}U_iU_i)}_{a}=\underbrace {\frac{\partial}{\partial x_j}(T_{ij}U_i)}_{b} 
			- \underbrace {T_{ij}S_{ij}}_{c}.
		\end{equation}
		\begin{itemize}
			\item a is transport of mean-flow energy.
			\item b represents transport of mean-flow energy by the stress $T_{ij}$. 
			\item c is called deformation work, and it represents kinetic energy of the mean flow that is lost to or retrieved from the agency 
			that generates the stress.
		\end{itemize}
		Here, $T_{ij} = -P\delta_{ij}+2\mu S_{ij} - \rho \overline{u_iu_j}$. So, Eq. \ref{enerEqMeanFlow1} can be rewritten as 
		\begin{equation}\label{enerEqMeanFlow2}
			\underbrace {\rho U_j\frac{\partial}{\partial x_j}(\frac{1}{2}U_iU_i)}_{a}=
			\frac{\partial}{\partial x_j}(
				\underbrace{-\frac{P}{\rho}U_j}_b + 
				\underbrace{2\nu U_iS_{ij}}_c - 
				\underbrace{\overline{u_iu_j}U_i}_d
			) +	\underbrace {2\nu S_{ij}S_{ij}}_{e} + \underbrace{\overline{u_iu_j}S_{ij}}_f. 
		\end{equation}
		\begin{itemize}
			\item a is transport of mean-flow energy.
			\item b is called pressure work.
			\item c is transport of mean-flow energy by viscous stresses.
			\item d is transport of mean-flow energy by Reynolds stresses.
			\item e is turbulence production.
			\item f is energy transport by turbulent motion.
		\end{itemize}
		Terms $c$ and $e$ in Eq. \ref{enerEqMeanFlow2} are negligible in most flows. This can be demonstrated easily by invoking 
		the scale relation as follow
		\begin{equation}
			S_{ij} \sim \frac{\partial U_i}{\partial x_j} \sim \frac{u}{l} \Rightarrow u \sim lS_{ij}, 
		\end{equation}
		\begin{equation}
			-\overline{u_iu_j} \sim u^2 \sim ulS_{ij}.
		\end{equation}
		So, the ratio of terms $d$ and $c$ can be estimated as
		\begin{equation}
			\frac{d}{c} \sim \frac{\overline{u_iu_j}U_i}{\nu U_iS_{ij}} \sim c\frac{ulU_iS_{ij}}{\nu U_iS_{ij}} \sim \frac{ul}{\nu}
			= Re.
		\end{equation}
		We see that $Re$ tends to be very large, so that $c$ could be negligible compared with $d$.
		Also the ratio of terms $f$ and $e$ can be estimated as
		\begin{equation}
			\frac{f}{e} \sim \frac{\overline{u_iu_j}S_{ij}}{\nu S_{ij}S_{ij}} \sim \frac{ulS_{ij}S_{ij}}{\nu S_{ij}S_{ij}}
			\sim \frac{ul}{\nu} = R_e, 
		\end{equation} 
		and $e$ can be negligible compared with $f$.
		\subsubsection{Pure shear flow}
		\subsubsection{The effect of viscosity}
	\subsection{Kinetic energy of the turbulence}
		The energy equation of turbulence reads
		\begin{equation}\label{turbEnerBuffet}
			\underbrace {U_j\frac{\partial}{\partial x_j}(\frac{1}{2}\overline{u_iu_i})}_a = -\frac{\partial}{\partial x_j}(
				\underbrace{\frac{1}{\rho}\overline{u_jp}}_b + 
				\underbrace{\frac{1}{2}\overline{u_iu_iu_j}}_c - 
				\underbrace{2\nu \overline{u_is_{ij}}}_d
			) - \underbrace{\overline{u_iu_jS_{ij}}}_e - \underbrace{2\nu \overline{s_{ij}s_{ij}}}_f.
		\end{equation} 
		\subsubsection{Production equals dissipation}
			In a steady, homogeneous, pure shear flow, Eq. \ref{turbEnerBuffet} reduces to
			\begin{equation}
				-\overline{u_iu_j}S_{ij} = 2\nu\overline{s_{ij}s_{ij}}.
			\end{equation}
			\begin{equation}
				\overline{s_{ij}s_{ij}} \gg S_{ij}S_{ij}.
			\end{equation}
		\subsubsection{Taylor microscale}
		\subsubsection{Scale relations}
		\subsubsection{Spetral energy transfer}
		\subsubsection{Further estimates}
	
	
	
	
	
	
\newpage
\appendix{APPENDIX A}	
\begin {center}
	\begin{table}[!htb]
		\centering
		\begin{tabular}{|c|c|c|c|c|}
			\hline
		 	& Length Scale& Velocity Scale & Time Scale & Reynolds number\\
			\hline
			molecular & $\xi$ & $a$ & $\frac{\xi}{a}$ & $R=\frac{a\xi}{\nu}=\frac{3}{2}$\\
			\hline
			laminar boundary layer & $l$ & $U$ & $\frac{l}{U}$ & $R=\frac{lU}{\nu}$ \\
			\hline
			turbulence boundary layer & $l$ & $u$ & $\frac{l}{u}$ & $R=\frac{lu}{\nu}\gg 1$\\
			\hline
			Kolmogorovmicroscales & $\eta$ & $v$ & $\tau$ & $R=\frac{\eta v}{\nu}=1$\\
			\hline
		\end{tabular}
		\caption{\label{scales} Length scales ,Velocity scales and Time scales.}
	\end{table}
\end{center}

\appendix{APPENDIX B}
	\begin{equation}
		u^2 \equiv \frac{1}{3}\overline{u_iu_i}
	\end{equation}
\end{document}
